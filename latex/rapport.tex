\documentclass{article}
\usepackage{lipsum}
\usepackage{graphicx}
\usepackage[swedish]{babel}
\usepackage[T1]{fontenc}
\usepackage[utf8]{inputenc}
\usepackage{hyperref}
\usepackage[margin=1in,left=1.5in,includefoot]{geometry}
\usepackage{listings}
\usepackage{color}
\usepackage{accsupp}

\definecolor{javared}{rgb}{0.6,0,0} % for strings
\definecolor{javagreen}{rgb}{0.25,0.5,0.35} % comments
\definecolor{javapurple}{rgb}{0.5,0,0.35} % keywords
\definecolor{javadocblue}{rgb}{0.25,0.35,0.75} % javadoc
\definecolor{backgroundcolor}{rgb}{0.95,0.95,0.95} % backgroundcolor

\lstset{
language=Java,
basicstyle=\ttfamily,
keywordstyle=\color{javapurple}\bfseries,
stringstyle=\color{javared},
commentstyle=\color{javagreen},
morecomment=[s][\color{javadocblue}]{/**}{*/},
backgroundcolor=\color{backgroundcolor},
numbers=left,
numberstyle=\tiny\color{black},
breaklines=true,
stepnumber=2,
numbersep=10pt,
tabsize=4,
showspaces=false,
showstringspaces=false,
frame=single}
\hypersetup{
    colorlinks,
    citecolor=black,
    filecolor=black,
    linkcolor=black,
    urlcolor=black
}
\usepackage{fancyhdr}
\pagestyle{fancy}
\fancyhead{}
\fancyfoot{}
\fancyfoot[R]{\thepage}
\renewcommand{\headrulewidth}{0pt}
\renewcommand{\footrulewidth}{1pt}
\begin{document}
\begin{titlepage}
    \begin{center}
        \line(1,0){300}\\
        \huge{\bfseries Rapport  }\\
        \line(1,0){200}\\
        [1cm]
        \textsc{\LARGE Examensarbete Alten Sverige AB }\\
        [13cm]
    \end{center}
    \begin{flushright}
        \textsc{\Large Max Jourdanis \\}
        max.jourdanis@gmail.com\\
        \# 0730474197\\
        [4pt]
        \textsc{\Large Henning Urrutia Målquist \\}
        malquisthenning@gmail.com\\
        \# 0730474197
        
    \end{flushright}
    \end{titlepage}
\tableofcontents 

    %Creates a new page
    \newpage
    
    %Removes the index on the page
    \thispagestyle{empty}

    %clears page after this
    \cleardoublepage

    %sets the pagenumber after this.
    \setcounter{page}{1}
    
\section{Inledning}
    \subsection{Bakgrund}
    Alten har en avdelning, Training, som arbetar med att sälja kurser främst
    till andra företag men även internt samt öppna kurser. I dagsläget sker
    planeringen av dessa kurser i Excel-dokument som skickas fram och tillbaka
    mellan Training och Invoice-avdelningen. Detta arbetssätt har varit rörigt
    och lätt att det blivit fel.\\
    Training har då önskat ett system som de ska kunna använda för att planera
    kurser med alla kostnader och intäkter, kommunicera med Invoice samt göra
    budgetar.\\
    Projektet påbörjades av lia-studenter 2017. Henning och Max blev tilldelad
    projektet under båda våra lia-perioder.\\
    Efter 10 veckor av vår andra lia-period blev vi dock tilldelade ett annat
    projekt så vi har totalt arbetat med detta i 20 veckor.\\
    Arbetet skedde agilt med kundmöten nästan veckovis. Under dessa möten visade
    vi upp vad som gjorts och diskuterade vad vis kulle fokusera på här näst för
    att så snabbt som möjligt nå en första version som skulle kunna användas. 
    \subsection{Mål}
        \subsubsection{Projektmål}
        Målet vi hade var alltså att få ut en första version av systemet som
skulle kunna användas. För detta krävdes Roller - De olika användarna av
systemet ska kunna göra olika saker. Invoice ska t.ex inte kunna komma åt eller
kunna ändra något i kurserna förutom att lägga till "workorder number". Totalt
ska det vara fem olika roller som alla har olika behörigheter. Säkerhet - För
att implementera roller ska även säkerhets biten ses över och delvis bytas ut.
Vi ska användas oss av Json Web Tokens. Hämta ut kunder och leverantörer från
internt system. I dagsläget görs skrivs allt detta manuellt och problem har
uppstått. Budget-vy - För att systemet ska vara användbart behövdes en ny
budget-vy. I denna vyn ska budgetar kunna sättas för de olika säljarna, för
öppna kurser samt en total budget. Här ska det också gå att se hur mycket av
budgeten som sålts, hur mycket som är kvar samt gross margin. Testmiljö - Sätta
upp en testmiljö där alla har tillgång att testa systemet och se att det
fungerar som tänkt. Buggar - t.ex så sparas datum fel och fel kurser visas i
kalendern. Se till att uträkningarna räknar på rätt saker. Namngivning - Reda ut
vad allt ska kallas. T.ex ska “utgifter” kallas “kostnader”Fakturerbara kurser -
Ska vara checkboxes som visar ifall en utgift eller kostnad är fakturerad
Pingfunktion - Det ska gå att pinga Invoice när ett kurstillfälle är redo för
fakturering samt ifall ett fakturerbart kurstillfälle behöver förändringar.

Kunden hade även önskningar för att arbetet i systemet skulle gå smidigare som
vi skulle göra i mån av tid. Sökning på kurser. För att arbetsflödet ska gå
smidigt behövs även en sökfunktion där det ska gå att söka på kursnummer, plats,
kursnamn samt datumintervall. Bifoga dokument. Detta ska gå att bifoga och ladda
ner dokument för varje kurstillfälle. Stilsättning - Utefter en UX-designers
rekommendationer. Varning i fall en med rollen invoice försöker lämna sidan där
den fyller i workordernummer och inte har meddelat training.

        \subsubsection{Slutmål}
    \subsection{Metod}
        \subsubsection{Verktyg}
            \subsubsection{Utvecklingsverktyg}
            \subsubsection{Programmeringsverktyg}
    \subsection{Kod}
        \subsubsection{Säkerhet}
        \begin{lstlisting}
@Override
protected void configure(HttpSecurity http) throws Exception {
    http.sessionManagement()
    .sessionCreationPolicy(SessionCreationPolicy.STATELESS);
    http.cors().and().csrf().disable().authorizeRequests()
    .antMatchers(HttpMethod.OPTIONS, "/**").permitAll()
            .antMatchers( "/*.*", "/assets/**", "/", "index.html", "**.js").permitAll()
            .antMatchers(HttpMethod.POST, "/api/admin/login").permitAll()
        //    .anyRequest().permitAll();
            .anyRequest().authenticated();
            http.apply(new JwtTokenFilterConfigurer(jwtTokenProvider));
}
            \end{lstlisting}
\end{document}